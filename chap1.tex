
\documentclass{book}

\usepackage{graphicx}

\usepackage{algorithm}
\usepackage{algorithmic}
\usepackage{amssymb}
\usepackage{amsfonts}
\usepackage{multirow}
\usepackage{amsmath}
\usepackage{algorithm}
\usepackage{algorithmic}
\usepackage{leftidx}% http://ctan.org/pkg/leftidx


\title{Numerical Methods for Robotics}

\author{ 
  Nicolas Mansard\\
        LAAS\\
        31077 Toulouse Cedex 4, France\\
 { \footnotesize Email: {\tt Nicolas.Mansard@laas.fr} }
}

\date{}

\newtheorem{definition}{Definition}
\newtheorem{proof}{proof}
\newtheorem{property}{Property}
\newtheorem{theorem}{Theorem}


\newcommand{\refchap}[1]{Cahpter~\ref{chap:#1}}
\newcommand{\refsec}[1]{Section~\ref{sec:#1}}
\newcommand{\eq}[1]{(\ref{#1})}
\newcommand{\mie}{\emph{i.e. }}
\newcommand{\meg}{\emph{e.g. }}

\newcommand{\EIN}[1]{\begin{equation}\label{eq:#1}}
\newcommand{\EOUT}{\end{equation}}
\newcommand{\EAIN}{\begin{eqnarray}}
\newcommand{\EAOUT}{\end{eqnarray}}
\newcommand{\BIN}{\begin{bmatrix}}
\newcommand{\BOUT}{\end{bmatrix}}

\newcommand{\repr}{\cong}

\newcommand{\mbf}[1]{\mathbf{#1}}
\newcommand{\derive}[2]{\frac{\deriv{#1}}{\deriv{#2}}}
\newcommand{\dpartial}[2]{\frac{\partial{#1}}{\partial{#2}}}
\newcommand{\ddpartial}[2]{\frac{\partial^2{#1}}{\partial{#2}^2}}

\newcommand{\traj}[1]{\underline{#1}}


\newcommand{\qdot}{\dot q}
\newcommand{\qddot}{\ddot q}



\begin{document}
\maketitle

\tableofcontents

% ------------------------------------------------------------------------------
\chapter{Outline of the class}

The objective of the class is to provide numerical methods to express, compute and reason about a motion to be executed by a physical platform. In particular, we will deal with representation of the motion, numerical resolution of algebraic equations and automatic numerical methods to approximate the solution to equality problems or minimization problems.

We will start with the static problem of finding one configuration answering to a set of constraints, typically finding a configuration such that the end effector of the robot is at a given position and orientation (chap 1: inverse geometry). This problem does not comprehend any concept of trajectory: only the final configuration is considered. In Chap. 2, we will consider the problem of finding the configuration velocity that brings the system closer to a given goal. This problem, named inverse kinematics, leads by integration to the resolution of the inverse geometry problem, and will be shown to be of a much simpler class of difficulty. Coming from geometry then kinematics, we will consider the system dynamics in the third chapter. In particular, we will present the inverse dynamics problem as an increment of the inverse kinematics problem. Finally, the last chapter we really consider the robot trajectory from an initial position to the final goal as a single object, and exhibit the methods to approximate the system optimality on the full trajectory. Compared to the initial static optimization problem, this last \emph{optimal control} problem is of a more complex class, but can be approximate to a static problem to the cost of a increase of dimensionality.

Each part of the class will be an oportunity to visit a general class of numerical resolution methods. In Chapter 1, we will introduce the Newton iterative optimization method. In Chapter 2, the pseudo-inverse will be the key tool. Chapter 3 we be the occasion to consider the optimization under constraints. Finally, in Chapter 4, we will quickly recall the optimal-control framework and consider more in detail the linear-quadratic regulator (LQR). 

% ------------------------------------------------------------------------------
% ------------------------------------------------------------------------------
% ------------------------------------------------------------------------------
\chapter{Inverse geometry}

We mainly consider in the part open kinematic chain, that is to say a tree of rigid bodies attached two by two by joints. Recall that the configuration space of the robot is a representation (desirably minimal) that uniquely defined the position of all the bodies of the robot. Before discussing the robot geometry, it is important to introduce the concept to manipulate a body in space. The correct structure to do so is the special Euclidean group $SE(3)$, which is described in \refsec{se3}.

The direct geometry function\footnote{The map that gives the position of the end effector with respect to the robot configuration is often called direct kinematics. The kinematics being the branch of mechanics that studies the \textit{motion} ($\kappa \iota \nu \eta \mu \alpha$, motion in Greek) of sets of points, \mie velocity, acceleration, we rather use direct geometry for the function $h$.} of the robot is the function that map one configuration of the robot to the corresponding placement (position and orientation) of the robot in the space. For open-chain robots, this function is very easy to compute. Performing an action with the robot often comes to finding one configuration where the end-effector is at the right place with the right orientation. Finding such a configuration is called the inverse geometry problem\footnote{This problem is often called \emph{inverse kinematics} in the litterature.}. 

In the general case, the inverse geometry problem is written as an algebraic problem, and solving it reduces to finding the roots (possibly several or a manifold of them) of a polynomials of several variables (one per degrees of freedom). In some particular cases, algebraic solutions exists to automatically or semi-automatically compute exact solutions or nearly exact solution to this problems. The most studied particular case has been the non-redundant manipulator with six degrees of freedom: in that case, we have the same number of constraints and variables. These methods are described in \refsec{anainvgeom}. 

Alternatively, numerical methods can be used to solve the problem in the general case, using descent methods. A basic introduction to the numerical methods is given in \refsec{numeric}, and their application to the inverse geometry problem is given in \refsec{numinvgeom}.

% ------------------------------------------------------------------------------
% ------------------------------------------------------------------------------
% ------------------------------------------------------------------------------
\section{Special Euclidean Group} \label{sec:se3}

\subsection{Euclidean space $\mathbb{E}^3$}

We consider the three-dimensional Euclidean space  $\mathbb{E}^3$. The Euclidean structure is totaly defined by selecting a representation $\mathbb{E}^3$, by selecting a coordinate frame, \mie an origin $\mathcal{O}$ and three independant direction $e_1,e_2,e_3$. In that case, any point $p \in \mathbb{E}^3$ can be identified with a vector $\mbf{p} \in \mathbb{R}^3$ which represents it:
\[ p \repr \mbf{p} = \BIN x \\ y \\ z \BOUT \in \mathbb{R}^3 \] 
where the symbol $\repr$ denotes the representation. In $\mathbb{E}^3$, a vector $v$ is determined by a pair of point $a$ and $b$. Its coordinates are defined in $\mathbb{R}^3$ by 
\[ v \repr \mbf{v} = \mbf{b} - \mbf{a} \]
where $\mbf{a},\mbf{b}$ are the representation of $a$ and $b$ in the chosen coordinate system.

The canonical basis on $\mathbb{R}^3$ implicitely defines a dot product $<v|w> \triangleq \mbf{v}^T \mbf{w}$ and a norm $|| v || = || \mbf{v}||_2 = \sqrt{<v|v>} = v_1^2 + v_2^2 + v_3^2$ with $v_1,v_2,v_3$ the components of $\mbf{v}$ ($\mbf{v} = (v_1,v_2,v_3)$).

Coming from the coordinate system, the cross product operator between two vectors $v,w$ is also defined by:
\[ v \times w  = \BIN v_2 w_3 - v_3 w_2 \\ v_3 w_1 - v_1 w_3 \\ v_1 w_2 - v_2 w_1 \BOUT \]
The cross product is an internal composition law of $\mathbb{R}^3$.

The transformations of $\mathbb{E}^3$ that preserves the dot and cross product are called the special Euclidean transformations (or rigid-body motion). The ``special'' attribute corresponds to the preservation of the cross product, \mie the orientation. The preservation of the dot product directly induces the preservation of the distances (both are equivalent using that $<u|v> = \frac{1}{4} ( ||u+v||^2 - ||u-v||^2)$). The set is denoted $SE(3)$:
\[ SE(3) \triangleq \Big\{ g : \mathbb{E}^3 \rightarrow \mathbb{E}^3, \textrm{ so that } \forall x,y \in \mathbb{E}^3, || g(x) - g(y) || = ||x - y|| \Big\} \]

The set $SE(3)$ equiped with the composition of application is a group (\mie the composition keeps the elements in the set, the 0 motion is the neural elements and any transformation is revertible). 

In the remaining, we will define some possible representations of $SE(3)$, \mie some vector spaces that surjectively maps $SE(3)$.


\subsection{Rotation $SO(3)$}

\subsubsection{Matrix representation}
We start with the subset of $SE(3)$ that keeps the origin fixed. This set is denoted $SO(3)$ for special orthogonal group. 
\[ SO(3) \triangleq \Big\{ g : \mathbb{E}^3 \rightarrow \mathbb{E}^3, \textrm{ so that } \forall x,y \in \mathbb{E}^3, || g(x) - g(y) || = ||x - y|| \textrm{ and } g(0) = 0 \Big\} \]
It can be shown that any elements $r$ of $SO(3)$ is a linear application in $\mbf{R}^3$: $SO(3)$ is a subgroup of the group of linear applications on $\mathbb{R}^3$ denoted by $GL(3)$. Any element of $SO(3)$ can then be represented by its action on the canonical basis, \mie its matrix in $\mathbf{R}^{3\times 3}$.

From the definition of $SE(3)$, we can easily show that the transformation of the canonical basis is still orthonormal, \mie that
\[ \forall i = 1..3, || r(e_i) || = 1 \]
\[ \forall i,j=1..3, i \neq j,  <r(e_i)|r(e_j)> = 0 \]
This defines six constraints, that can be summarized by 
\EIN{rtr} R R^T = I \EOUT
where $R = [ r(e_1) r(e_2) r(e_3) ]$ is the matrix representation of $r$. In addition, since the orientation (cross product) is preserved, we also have the positivity of the determinant, \mie:
\[ det(R) = +1\]

\subsubsection{Angular velocities $\mathfrak{so}(3)$}

Consider now a curve $r(t)$ in $SO(3)$ represented by $R(t)$ in $\mathbb{R}^{3\times3}$. By derivation of \eq{eq:rtr}, we have that:
\[ \dot{R}(t) R(t)^T = - ( \dot{R}(t) R(t)^T )^T \]
Say otherwise, $\dot{R}(t) R(t)^T$ is antisymmetric. The group of antisymmetric matrices is denoted $\mathfrak{so}(3)$:
\[ \mathfrak{so}(3) \triangleq \Big\{ \BIN 0 & -w_3  & w_2 \\ w_3 & 0 & -w_1 \\ -w_2 & w_1 & 0 \BOUT, (w_1,w_2,w_3) \in \mathbb{R}^3 \Big\} \]
The group $\mathfrak{so}(3)$ is evidently isomorph to $\mathbb{R}^3$. We denote $w \in \mathbb{R}^3 \rightarrow \hat{w} \in \mathfrak{so}(3)$ the isomorphism from $\mathbb{R}^3$ to $\mathfrak{so}(3)$ and $\check{.}$ its reciprocal. The vector $w(t)$ can be identified to the angular velocity corresponding to $r(t)$.

Consider now the curve obtained by integration of a constant velocity $w \in \mathfrak{so}(3)$:
\[ \forall t>0, \dot{R}(t) = \hat{w} R(t), R(0) \in SO(3) \]
The differential equation is uniquely integrable in $\mathbb{R}^3$ (theorem of Cauchy). The expression of the integrale is obtained using the exponential of $\hat{w}$ defined by its power serie:
\[ exp( t \hat w  ) = e^{t \hat w} = \sum_{n=0}^{+\infty} \frac{1}{n!} (t \hat w)^n \]
Using this definition, the integrale is:
\[ R(t) = R(0) e^{t \hat w} \]
It is easy to prove that the exponential satisfies the differential equation, and its uniqueness is ensured by the theorem of Cauchy.

From the differential construction, we obtain that $e^{t \hat w}$ is a rotation matrix. This property can also be proved directly, by verifying that ${e^{t \hat w}}^T e^{t \hat w} = I$ using the power serie, and by showing that $det(e^{t \hat w})=1$ by continuity from $det(e^0)=1$.

\subsubsection{Rodrigues's formula}

The infinite power serie of the exponential map easily reduces to a simple formula using the simple structure of $\mathfrak{so}(3)$. Indeed, we have the two following properties:
\[ \hat{w}^2 = ww^T - I_3 \textrm{ and } \hat{w}^3 = -\hat{w} \]
The development immediatly reduces to:
\[ e^{t \hat w} = I + sin(t) \hat{w} + \frac{1-cos(t)}{t} \hat{w}^2 \]
using the serie of $\sin(t) = \sum \frac{(-1)^{n}}{(2n+1)!} t^{2n+1} $ and $\cos(t) = \sum \frac{(-1)^2}{(2n)!} t^{2n}$.
This expression is named the Rodrigues's formula.

The exponential map is an application from $\mathfrak{so}(3)$ to $SO(3)$:
\[ exp: w \in \mathfrak{so}(3) \rightarrow e^{\hat{w}} \]
The cases studied above is only decoupled expression, $w$ being the direction and $t$ the magnitude of the rotation. From the Rodrigues's formulat, we directly see that the exponential is not injective, since the norm can be chosen modulo $2\pi$.


\subsubsection{Canonical exponential coordinates}

The exponential map from $\mathfrak{so}(3)$ generates rotation matrices. It can be shown that the map generates all the group $SO(3)$, that is to say that it is surjective (onto). This is done by building the reciprocal, called the logarithm map. Being given a rotation matrix $R$ with $r_{ij}$ its components, the logarithm is defined by:
\[ log : R\in SO(3) \rightarrow w \in \mathfrak{so}(3) \]
\[ || w || = cos^{-1} ( \frac{\textrm{trace}(R) -1}{2}) , \quad \frac{1}{||w||} w = \frac{1}{2\sin(||w||)} \BIN r_{32}-r_{23} \\  r_{13}-r_{31} \\  r_{21}-r_{12} \BOUT \]

The surjectivity guarantees that any rotation matrix $R$ can be generated by taking the exponential of a vector of $\mathbb{R}^3$. The group $SO(3)$ can therefore be represented using $\mathbb{R}^3$. This is called a parametrization of the group. We call this parametrization the canonical exponential coordinates.

Compared to the representation using rotation matrix, of dimension $9$ but with six constraints on the effect on the canonical basis, this representation is of dimension 3 without any constraint. It is therefore minimal. 

Intuitively, the coordinates $w = \theta u$ with $\theta = || w ||$ and $u = \frac{1}{\theta} w$ corresponds to the integration of an unitary rotation velocity during a time $\theta$, or said differently to a rotation of an angle $\theta$ around the direction $u$. It corresponds to the older results that any rotation of a solid in space can be brought back to a pure rotation around a particular axis. The coordinates are also named ``angle vector'' or ``u theta'' for this reason.


\subsection{General overview: standard Lie Group representation}

The definition of the canonical coordinates seems pretty artificial, since we had to introduce a complex map coming from an arbitrary differential equation. In fact, the process to obtain the parametrization is coming from the very topological structure of the group. The same process can be used for obtaining usefull parametrization on similar topologies. We make here a brief overview of the topological tools that where used, in the general case.

\subsubsection{Differential manifold}
A differential manifold $\mathcal{M}$ is a set that is locally diffeomorphic to a vector space $\mathbb{R}^n$, that is to say, in any point $x \in \mathcal{M}$, there exists a local neighborhood $\mathcal{U}$ of $x$ with a diffeorphism $\psi$ from $\mathcal{U}$ to a neighborhood of $0_n$ in $\mathbb{R}^n$. A collection of $\mathcal{U},\phi$ covering $\mathcal{M}$ is called an atlas of $\mathcal{M}$. Each $\phi$ is called a local coordinates system. We consider here only cases where all the atlas are equivalent, \mie the diffemorphism are kept from one local coordinates system to the other.

Intuitivelly, a differentiable manifold is a set that locally behaves like $\mathbb{R}^n$.

\subsubsection{Tangent space}
On each point $p$ of a differential manifold $\mathcal{M}$ we write $C^\infty(p)$ the set of all smooth function from any neighborhood of $p$ into $\mathbb{R}$. The tangent space $T_p(\mathcal{M})$ is the subset of linear form over $C^\infty(p)$ satisfying the ``derivation'' Leibniz rule, \mie:
\EAIN
T_p(\mathcal{M}) \triangleq \Big\{ &X_p : f \in C^\infty(p) \rightarrow \mathbb{R},  \\ & \forall f,g \in C^\infty(p), \alpha \in \mathbb{R}, X_p(\alpha f + g ) = \alpha X_p(f) +X_p(g) \\ &\textrm{ and } X_p(fg) = X_p(f)g + f X_p(g) \Big\} \EAOUT
Intuitively, the tangent space defines all the possible directions at which a tangent to a curve of $\mathcal{M}$ in $p$ can pass.

The tangent space is a linear space. A basis can be built from the canonical basis $(x_1,...,x_n)$ of $\mathbb{R}^n$ using any local coordinates $\phi$ at $p$. The basis is denoted $(\dpartial{}{x_1}, ... , \dpartial{}{x_n} )$.

A vector field $X$ is simply the association of a vector of the tangent space to any point of the manifold:
\[ X: p \in \mathcal{M} \rightarrow X_p \in T_p \mathcal{M} \]

\subsubsection{Lie group and algebra}

A Lie group $\mathcal{G}$ is a differential manifold with a smooth group operation and inverse.  The neutral element  of the group is denoted by $1_\mathcal{G}$.
\newline\emph{In the case of spatial rotation, the Lie group is $SO(3)$}. \medskip

Using the group law, invariant vector fields can be produced by morphing a vector of the tangent space at $1_\mathcal{G}$ by all the left composition $L_g : h \in \mathcal{G} \rightarrow hg$. Details can be found in [Murray 93]. Therefore, the tangent space at $1_\mathcal{G}$ is isomorphic to the set of invariant vector fields. An invariant vector field is denoted by $X_w$ with $w \in 1_\mathcal{G}$ the generator.

This space $T_1 \mathcal{G}$  is called the standard Lie algebra associated with the Lie group and denoted by $\mathfrak{g}$. It inherits its vector-space structure from the tangent space, to which is added the algebraic structure coming from the Lie bracket operation, inherited from the vector field set. This operation being anectodic for this study, it is not described forward.
\newline\emph{In the case of spatial rotation, the Lie algebra is $\mathfrak{so}(3)$}. \medskip

From the standard Lie algebra, it is possible to define the exponential map from $\mathfrak{g}$ to a neighborhood of $1_\mathcal{G}$.  For any $w \in T_1 \mathcal{G}$, let $g_w: t \in \mathcal{R} \rightarrow g_w(t) \in \mathcal{G}$ denotes the integral curve of the invariant vector field $X_w$ generated by $w$ and passing through $1_\mathcal{G}$ at t=0. The exponential map of $w$ is defined from $g$ after unitary integration by $exp(w) = g_w(1)$, or for any $t$:
\[ \exp(t w) = g_w(t) \]
\emph{In the case of spatial rotation, the exponential map definitions fit}. \medskip

The exponential map is a diffeomorphism from a neighborhood of $0_\mathfrak{g}$ to a neighborhood of $1_\mathcal{G}$. This defines a local coordinate system around $1_\mathcal{G}$. Using the group law of $mathcal{G}$, this local coordinate system can be morphed to a proper atlas.

Moreover, if $\mathcal{G}$ is compact, the exponential map is surjective. In that case, the exponential map defines a global coordinate system, inherited from the vector space structure of $\mathfrak{g}$. It is called the canonical exponential coordinates.

\subsection{The exponential map is not everything}

Despite the topological origin, the exponential coordinates do not keep the topology of the initial space. In particular, the exponential map is not injective in general, which means that several coordinates correspond to one same point. Typically for $SO(3)$, the exponential coordinates are defined modulo $2 \pi$ and does not keep the symmetry of the group with a singularity in the neighborghood of the null rotation $I_3$ (the direction being degenerated when the angle is becoming null).

\subsubsection{Euler angles}
Other representation of $SO(3)$ exists. The first to be known are the Euler angle, which corresponds to a sequence of three elementary rotations of angle $a_i$, $i=1..3$, around three axes. Two types can be distinguished: the first when the rotations are performed around fixed axes. They can be written using the exponential map by:
\[ r(a_1,a_2,a_3) = e^{a_1 \hat{w_1}} e^{a_2 \hat{w_2}} e^{a_3\hat{w_3}} \]
where $w_i$, $i=1..3$ are three rotation axis.
The second type correspond to the rotations that are performed around axes that rotates with the transformation. In that case:
\[ r(a_1,a_2,a_3) = e^{a_1 \hat{w_1} + a_2 \hat{w_2} + a_3\hat{w_3}} \]
Note by the way that $e^{\hat w_1 + \hat w_2} \neq e^{\hat w_1}e^{\hat w_2}$ except in particular cases.

The most used rotation are the roll-pitch-yaw. It corresponds to the rotation of the first type, with $w_3 = (1,0,0)$ (roll, around the $X$ axis), $w_2 = (0,1,0)$ (pitch, around the $Y$ axis) and $w_1 = (0,0,1)$ (yaw, around the $Z$ axis). These angles are much used in the aerospace industry.

\subsubsection{Quaternions}

The quaternion are a way to get rid of the singularity of the exponential coordinates in 0 and of their non surjectivity modulo $2\pi$. Geometrically, the normalized quaternions correspond to the sphere $\mathbb{S}^3$ which is a submanifold of $\mathbb{R}^3$, while keeping the same topology than $\mathbb{S}^2$ for $\mathbb{R}^2$. They are defined from the complex numers $\mathbb{C}$ exactly like the complex numbers are defined from $\mathbb{R}$ using $\mathbb{C} = \mathbb{R} + \mathbb{R}i $, with $i^2 = -1$. We define the imaginery number $j$ such that $j^2 = -1$ and $ij = -ji$. By convenience, we denote by $k = ij$, that has the properties that $k^2=-1$, $jk = i$ and $ki=j$. The quaternion space $\mathbb{H}$ is then defined by:
\[ \mathbb{H} = \mathbb{C} + \mathbb{C} j 
= \mathbb{R} + \mathbb{R} i + \mathbb{R}j + \mathbb{R} k \]
The quaternions are equiped with the multiplicative law of $\mathbb{R}$. It is relatively easy to show that the inverse elements in $\mathbb{H}$ is:
\[ q^{-1} \triangleq \frac{\overline{q}}{||q||^2} \]
with the  conjugate of the quaternion $q=q_0 + q_1 i + q_2 j + q_3 k$ being $\overline{q} \triangleq q_0 - q_1 i - q_2 j -q_3 k$ and the norm being $||q||^2 = q \overline{q}$, that is to say the Euclidean norm of $\mathbb{R}^4$.

The quaternion space has many intersting property. The interest for the Euclidean motion representation is that it can embed the rotation group $SO(3)$ in its unitary sphere $\mathbb{S}^3$:
\[ \mathbb{S}^3 \triangleq \left\{ q \in \mathbb{H}, ||q|| = 1 \right\} \]
The sphere $\mathbb{S}^3$ is trivially a subgroup of $\mathbb{H}$. It is directly associated with $\mathfrak{so}(3)$ while preserving the group structure of $SO(3)$ using the simple map:
\[ (\theta,u) = (||w||,\frac{1}{||w||}) \repr r \in SO(3) \rightarrow q(r) = cos(\frac{\theta}{2}) + sin(\frac{\theta}{2})(u_1 i + u_2 j + u_3 k) \]
with $u = (u_1,u_2,u_3)$ the coordinates of $u$. It is possible thus teddious to verify that the group structures of $SO(3)$ and $\mathbb{H}$ fit. 

The reciprocal in the proper subspace of $\mathfrak{so}(3)$ is:
\[
\theta = 2 cos^{-1} (q_0),
\quad u = \left\{  \begin{array}{ll} 0, & \theta = 0, \\ \frac{1}{sin(\theta/2)} q_{1:3} & \theta \neq 0 \end{array}\right. \]
with $q_{1:3} = (q_1,q_2,q_3)$. The conditional equality depending of $\theta=0$ corresponds to the singularity of $\mathfrak{so}(3)$ around the zero rotation. Moreover, we can notice that both $q$ and $-q$ produce the same rotation. $SO(3)$ in fact corresponds topologically of the half sphere or projection plane $RP^3$.

\subsubsection{Summary of rotation representations}
The quaternion representation is of dimension $4$ with one constraint ($||q||=1$). It is therefore not minimal. However, it more adequatly captures the topology of $SO(3)$. In addition, it is equiped with a composition law corresponding to the composition of rotation, while being cheaper to compute numerically. Finally, the action of the linear map $q \repr r \in SO(3)$ on any $v \in \mathbb{R}^3$ can be computed directly from the quaternion using:
$ r(v) = q \tilde{v} q  $
where $\tilde v = v_1 i + v_2 j + v_3 k$. In many situations, the quaternions are the most adapted representation of $SO(3)$. The exponential coordinates have to be chosen when minimality is crucial, while $\mathbb{R}^{3\times3}$ might be more interesting when many vector multiplications are computed (typically when rotating a cloud of thousand of points).

\subsection{Rigid displacement $SE(3)$}

We now come back to the initial $SE(3)$ group, the set of all application in $\mathbb{E}^3$ that preserves the distance and the orientation. $SE(3)$ can be shown to be the composition of a pure rotation $r \in SO(3)$ and a pure translation $t \in \mathbb{R}^3$:
\[ SE(3) = \mathbb{R}^3 \times SO(3) = \{ (r,t), r\in SO(3), t \in \mathbb{R}^3 \} \]

\subsubsection{Homogeneous matrix representation}
Denoting by $m \repr (r,t)$ a rigid motion of $SE(3)$, its application on a point $p \in \mathbb{E}^3$ is:
\[ m(p) = r(p) + t \]
The application $m$ is therefore affine in $\mathbb{R}^3$. In can be transformed for convenient to a linear map to the cost of embeded it in the larger space $GL(4) \repr \mathbb{R}^{4\times4}$ by:
\[ M p = \BIN R & t \\ 0_{1\times3} & 1 \BOUT \BIN p \\ 1 \BOUT \]
where $M \in \mathbb{R}^{4\times4}$ is the homogeneous representation of $m \in SE(3)$, $R \in  \mathbb{R}^{3\times3}$ is the matrix representation of the rotation part of $m$, $t \in \mathbb{R}^3$ is the translation part of $m$ and $p \in \mathbb{R}^3$ is the vector representation of the point $m$ is applied to.

The matrix product fits to the composition operation of $SE(3)$, providing directly the group structure of $SE(3)$:
\begin{itemize}
\item if $m_1,m_2 \in SE(3)$, then $m_1 \circ m_2 \in SE(3)$. Moreover, the law is associative.
\item the identity of $SE(3)$ corresponds to $I_4$.
\item the inverge $m^{-1}$ of $m$ is represented by:
\[ M^{-1} = \BIN R^T & -R^T p \\ 0 & 1 \BOUT \]
\end{itemize}

\subsubsection{Pose-u-theta coordinates}

Like for $SO(3)$, the matrix representation of $SE(3)$ is convenient but not minimal: 16 components are necessary but are subject to 7 constraints, three for the $R$ part and 4 for the last row (these 4 last components are indeed trivial). Moreover, it is more convenient to express the coordinates as a vector space $\mathbb{R}^n$ ($n$ being intuitively equal to 6), where the neutral element is $0_n$ and where the group topology is closer to $\mathbb{R}$ than $GL(4)$.

Using the canonical exponential coordinates of $SO(3)$, we can directly propose the coordinates system $\mathbb{R}^3 \times \mathfrak{so}(3)$:
\[ m = (r,p) \repr \BIN p \\ w \BOUT \]
where $r \repr w$ is the exponential representation of the rotation part of $m \in SE(3)$. Like $\mathfrak{so}(3)$, this coordinates system has no group structure that fits the topology of $SE(3)$.

\subsubsection{Exponential coordinates}

This intuitive coordinates system is close to be the exponential coordinates of $SE(3)$ but is not exactly it. This means that their derivative does not fit with the tangent space to $SE(3)$, \mie to the velocity of rigid bodies. To obtain the exponential coordinates of $SE(3)$, the procedure is similar to the one performed in $SO(3)$. 

Let $m(t)$ be a curve in $SE(3)$. By analogy, consider the following matrix:
\[ \dot{M}(t) M(t)^{-1} = \BIN \dot{R}(t) R(t)^T & \dot{p}(t) - \dot{R}(t) R(t)^T p(t) \\ 0&1 \BOUT \]
We denote by $w(t)$ the angular velocity $\hat{w}(t) = \dot{R}(t) R(t)^T$. The last columns is then denoted by $v(t) = \dot{p}(t) + p(t) \times w(t)$. The pair $\nu(t) = (v(t),w(t)) \in \mathfrak{se}(3) = \mathbb{R}^3 \times \mathfrak{so}(3)$ can be identified to the tangent vector to the curve $m(t)$. It corresponds to the kinematic screw (``torseur cin\'ematique'' in French) of the rigid body moving following $m(t)$, expressed at the origin of the coordinates system of $\mathbb{E}^3$ in which $m(t)$ is expressed. See next subsection for details.

By integration of a constant kinematic screw $\nu = (v,w)$ on $[0,1]$, the exponential map is obtained.
\[ e^{\nu} =
\left\{ \begin{array}{cl}
 \BIN e^{w} &  \frac{1}{||w||}\Big( ww^Tv + (I-e^{w}) w \times v \Big) \\ 0 & 1 \BOUT,& \textrm{if } ||w||>0 \\
 \BIN I_3 &  v \\ 0 & 1 \BOUT,& \textrm{otherwise}
\end{array}\right.
\] 

Intuitively, the kinematic screw can be understood has representing a ``screw'' motion, \mie a pure rotation of angle $\theta$ around a fixed axis in $\mathbb{E}^3$ followed by a pure translation of length $d$ in the direction of this same axis. Such a screw motion is defined by six parameters (5 for the axis \meg 3 for a point $x$ on the axis and 2 for a  direction (normalized vector $u$) plus 1 for the pitch \mie the ration $h=d/\theta$ between translation and rotation, $d=\infty$ being a pure translation). Indeed, if we set:
\EAIN u \triangleq \frac{1}{||w||} w  \\ x \triangleq u \times v \\ h \triangleq w^T v \EAOUT
we obtain an equivalence (isomorphism) between the kinematic screw and the screw motion. This isomorphism prooves by the way the theorem of Chasles stating that any motion on a rigid body can be expressed as the motion of a screw, for a uniquely defined screw (that may typically not be attached to the rigid body but to an imaginary point).
See [Murray 94] for a detailed discution.

The exponential map is sujective into $SE(3)$. The inclusion of its image in $SE(3)$ is straightforward from its matrix structure. The covering is ensured by the compactness of $SE(3)$ but can also be constructively demonstrated by building the logarithm map:
\[ log: m = (r,p) \rightarrow (w, v) \]
with $w=log(r)$, $v = p$ when $||w||=0$ and $v = ||w|| \Big( ww^T + (I-e^w)\hat{w} \Big)^{-1} p$ otherwise.

\subsubsection{Movement of coordinates system in $SE(3)$}

Any rigid movement $m \in SE(3)$ displaces a coordinates system of $\mathbb{E}^3$ (\mie a point $\mathcal{O} \in \mathbb{E}^3$ and three orthonormal vectors of $\mathbb{R}^3$) into another coordinates system. Consider two coordinates systems of $\mathbb{E}^3$ denoted by their frames $\mathcal{F}_a$ and  $\mathcal{F}_b$. The coordinates of a point $p\in \mathbb{E}^3$ are denoted by $^ap$ in $\mathcal{F}_a$ and by $^bp$ in $\mathcal{F}_b$. We denote by $^am_b \repr \leftidx{^a}{M}{_b} \in \mathbb{R}^{4\times4}$ the rigid motion displacing the frame $\mathcal{F}_a$ into the frame  $\mathcal{F}_b$. In that case, we have:
\[ ^ap = \leftidx{^a}{M}{_b} \leftidx{^b}{p}{} \]
The notation follows the tensor notations. 

Consider now a fixed inertial frame $\mathcal{F_o}$ and a body frame $\mathcal{F}_b$ attached to a moving rigid object, such that the two frames match at $t=0$. The  timeof the rigid body position of any point $p$ through time can be described by only giving the trajectory $^om_b(t)$ in $SE(3)$:
\[ o^p(t)  = \leftidx{^o}{M}{_b}(t) p_b\]
where $p_b = p_b(0)$ are the coordinates of the fixed position of the point $p$ in the body frame.

We say the $^om_b$ is the placement (\mie position and orientation) of the body $b$ in the coordinates system $o$.

\subsection{A first taste of rigid velocities}

% ------------------------------------------------------------------------------
% ------------------------------------------------------------------------------
% ------------------------------------------------------------------------------
\section{Direct geometry} \label{sec:directgeom}

In the previous section, we have seen how rigid displacements can be represented. We now consider a more complex class of movements: the articulated rigid motion, \mie the considered system is composed of a finite set of rigid elements, each of them moving subject to some given  constraints imposed by the joints linking them. We consider here only the case of open kinematic chain (properly describe below described) and will discuss only quickly the case of kinematic loops later.  Like for $SE(3)$, we will discuss here the possible solution to numerically represent the motion of such systems and to compute the motion (position, velocities, etc) of each points of the system.

The most difficult part of the study has been done with $SE(3)$. This section is thus notably shorter.

\subsection{Kinematic tree}

We consider the graph structure describing the links between each rigid body: each node of the graph represents one rigid body and the edges of the graph represents the joints connecting two bodies together. A body can be connected to several other bodies through several joints. It is not connected to itself and only connected once to any other. Open kinematic chains are those for which there is no loops in the graph, \mie it is possible to selec a tree structure for the kinematic graph. Simple kinematic chain are reduced to a trivial tree (only one branch, each node has only one son and one father at maximum).

Two cases can be considered, that are mathematically the same but leads to two different interpretations of the root: the robot can be attached to the ground. In that case, the root of the tree, named the base link, is fixed. Or the robot can be free to move. The root of the tree is then arbitrarily selected. We say that the robot has a free flyer.
Both case are similar in the sense that a free-flyer robot can be seen as attached to a fixed ``ground'' body by a ``free-flyer'' joint; or alternatively a fixed robot is a free-flyer chain with the base body having infinite inertia.

In the following, we take the first convention: the base link is fixed, and a free-flyer robot is obtained by adding an artificial ``free-flyer'' link.

\subsection{Bodies}

Each body, represented by a node of the graph, is a rigid set of points. The body is associated with a body coordinates system, into which any point is described by a constant vector of $\mathbb{R}^3$. 

\subsection{Joints}

The joints are the moving pieces of the articulated rigid body. Each joint is attached two joints, one being the father $i$ and the second the son $i+1$. In general, the configuration $q_i$ of a joint is an element of a differentable manifold $\mathcal{Q_i}$ that uniquely defines the placement of the son body in the coordinates system of the father body:
\[ K_i: q_i \in \mathcal{Q_i} \rightarrow \leftidx{^a}{m}{_b}(q) \in SE(3) \]
For convenience, we generally rather consider a class of joint, caracterized by their $K_i : \mathcal{Q_i} \rightarrow SE(3)$ function. The joint linking body $i$ to body $i+1$ is then defined by its placement in the system $i$ along with its kinematic function $K_i$.

In all the relevant cases, $K_i$ is a smooth function and $Q_i$ a compact Lie group. It is therefore possible to associate a global vector representation to $Q_i$ and a Jacobian matrix to the tangent application to $K_i$. The standard position of the joint is given by the neutral element of the group $1_\mathcal{Q}$, expecting $K(1_\mathcal{Q}) = 1_{SE(3)}$.

The relative placement of body $i+1$ in the coordinate system of body $i$ is then given by:
\[ q_i \in \mathcal{Q_i} \rightarrow \leftidx{^i}{m}{_{i+1}}(q_i) = \leftidx{^i}{m}{_{i+1}^0} \  K_i(q_i) \]
with $\leftidx{^i}{m}{_{i+1}^0} = \leftidx{^i}{m}{_{i+1}}(1_\mathcal{Q}$ the placement of the joint in the coordinates system of body $i$.


\subsubsection{Revolute joint}

The most classical joint is the revolute joint: one degree of rotation around a fixed axis, typically the $Z$ axis. The configuration space of the joint is then $\mathcal{S}^1$ the unitary circle. Most of the time, the joint angle is bounded by an upper and a lower joint limit. The configuration space is then an interval of $\mathbb{R}$. The kinematic function $K$ is simply:
\[ K_i: q \in \mathbb{S}^1 \rightarrow \BIN \cos q & -\sin q & 0 & 0 \\ \sin q & \cos q & 0 & 0 \\ 0 & 0 & 1 & 0 \\ 0 & 0 & 0 & 1 \BOUT \]

\subsubsection{Prismatic joint}

Prismatic joint can translate (typically in an interval) along a given fixed axis, say $Z$. The configuration space is direcly $\mathbb{R}$. The kinematic function $K$ is trivially $K_i(q) = \big(I,(0,0,q)\big) \in SO(3)\times\mathbb{R}^3$.

\subsubsection{Free-flyer joint}

Free-flyer joints are configured by $SE(3)$ with their kinematic function being the identity on $SE(3)$. 

\subsubsection{Other joints}

Other joints can be found, even if very rarely used in robotics. Sperical joints allow free and symetrical rotations. Their configuration space is $SO(3)$, with their kinematics being $r \rightarrow (r,0_3)$. Very few efficient mechanisms exist to actuate such a joint. A similar effect is rather used by a sequence three revolute joints with concurent rotation axes. However, spherical joints are very often used for modeling purpose, for example for the model of the human body (typically, the hip joint), or for passive mechanisms.

The planar joint is a free 2D motion: two translations, one rotation. Its configuration space is $SE(2)$. The cylindric and helicoidal joints are a composition of on translation and one rotation with shared axis, the first one with free , the second with coupled rotation and translation motions.

\subsection{Configuration space}

The configuration space of the robot is typically the result of the Cartesian product of the configuration space of all the joints. Its reprensation is the cartesian product of the representation of the joint configuration spaces. The order of the terms in the Cartesian product only matters for the representation of the configuration space. It is purely arbitrary.
In the case of a simple kinematic chain, the only relevant order is the one given by the chain.
\[ \mathcal{Q} = \times \mathcal{Q}_i \]
For kinematic trees, the selected order should be compatible by the implicite partial order defined by the tree. The order in which the branches are explored is typically arbitrary and has to be documented.

When only minimal vector representation of the $\mathcal{Q}_i$ are used, a minimal vector representation of $\mathcal{Q}$ is obtained. The dimension of the vector $q$ is equal to the dimension of the manifold $\mathcal{Q}$.

\subsection{Direct geometry function}

We called the end effector of the robot one of the body of arbitrary importance (typically the one carrying the tool). 
For all this part, we can consider only simple chain without significant loss of generality. In that case, the end effector is the last body of the chain.

The robot direct geometry function (as said upper, often named direct, or forward, kinematic function in the litterature) is the function that maps the configuration space to the placement of the end effector in $SE(3)$:
\begin{align*}
 K: q = (q_0, ... , q_{n-1}) \in \mathcal{Q}  \rightarrow & \prod_{i=0}^{n-1} \leftidx{^i}{m}{_{i+1}^0} K_i(q_i) \\
&=  \leftidx{^0}{m}{_{1}^0} K_0(q_0) \leftidx{^1}{m}{_{2}^0} ...   \leftidx{^{n-2}}{m}{_{n-1}^0} K_{n-1}(q_{n-1})
\end{align*}

The direct geometry function $K$ can be represented as a function from the vectorial representation of $\mathcal{Q}$ to $\mathbb{R}^{4\times4}$.
In the case of all the joints that we considered upper, the components of the homogeneous matrix are polynoms of the components of $q$ and of sinus and cosinus of the components of $q$, with the coefficients of the polynoms depending on the placement of the joints in their reference body. The degrees of the polynoms are equal to the dimension of $q$.

The notion of end effector is arbitrary. In particular, the end effector can be a sub part of the last body. In that case, a last constant rigid move $\leftidx{^{n-1}}{m}{_n}$ is added to the geometry function:
\[ K(q) =  \leftidx{^0}{m}{_{1}^0} K_0(q_0) \leftidx{^1}{m}{_{2}^0} ...   \leftidx{^{n-2}}{m}{_{n-1}^0} K_{n-1}(q_{n-1}) \leftidx{^{n-1}}{m}{_n} \]

In the case of kinematic trees, several direct geometry functions are defined for each end effector attached to the end of each leaf of the tree.

\subsection{Denavit-Hartenberg parametrization}

The Denavit-Hartenberg method is a minimal parametrization of a revolute-prismatic kinematics chain. It is used to minimaly describe the relative joint placements $\leftidx{^{i}}{m}{_{i+1}^0}$. This parametrization was a de facto standard 10 years ago and is still often used (even if, now, the $SE(3)$ displacement is generally encoded directly using a standard representation).

Indeed, the static displacement $\leftidx{^{i}}{m}{_{i+1}^0}$ requires at least six parameters to be stored. However, the choice of the rotation around the axes of the joint $i$ and of the the joint $i+1$ are \emph{free} (translation in case of prismatic joints), in the sense that any static angles around these axes can be compensated to the cost of a translation of the joint configuration interval. The Denavit-Hartenberg parametrization therefore only needs 4 parameters to encode $\leftidx{^{i}}{m}{_{i+1}^0}$. 

The detail of how the static displacement can be computed from the Denavit-Hartenberg parameters can be found in [Murray93].

\subsection{Workspace}

The workspace $\mathcal{W}$ of the robot is the image space of the direct geometry function, \mie:
\[ \mathcal{W} = \Big\{ m \in SE(3), \textrm{ so that } \exists q \in \mathcal{Q}, K(q) = m \Big\} \]

In general, $\mathcal{W}$ is a compact non-convex subset of $SE(3)$. Its border are most of the time very difficult to compute explicitely.

% ------------------------------------------------------------------------------
% ------------------------------------------------------------------------------
% ------------------------------------------------------------------------------
\section{Analytical inverse geometry} \label{sec:anainvgeom}

The direct geometry maps the configuration space to the placement of the end effector. The map is in general not injective (several configuration corresponds to the same placement) and definitely not surjective: the robot workspace is not the entire world. However, programming a robot often comes back to placing its end effector to a given reference. The inverse geometry problem is defined by: given a reference placement in $SE(3)$, find one or all the configurations that matches this reference. The inverse maps $K^+$ the workspace to some non-empty  possibly non-trivial subset of the configuration space:
\[ K^+: m \in \mathcal{W} \rightarrow K^+(m) \subset \mathcal{Q} \]

The typicaly case of study is the non-redundant case: consider a robot with six joints. The problem is then properly balanced, with 6 constraints and 6 variables. 

In general, we do not know how to compute the inverse map explicitely. This was the subject of very extensive works of research in the 90's, that produce some efficient solutions in the case of non-redundant robot manipulator  $6 \times 6$ problem. These approaches where then extended to similar case, like in the case of one degree of redundancy (7 joints, 6 placement constraints).

\subsection{Overview of the possible methods}

The following description of the state of the art is taken from Juan Cort\`es PhD thesis [Cortes03].

\subsubsection{Substitution: geometrical view}

In most industrial applications, mechanisms are normally designed with
particular geometries which allow a closed-form analytical solution to the loop closure
equations. For instance, non-redundant serial manipulators often have the last three
revolute axes intersecting in a same point, which greatly simplifies the solution of the
inverse kinematics problem [Angeles 03].

\subsubsection{Substitution: algebraic view}

Two main approaches have been classically used: continuation and elimination (see [Nielsen 97] for a complete survey of these techniques). Polynomial continuation methods [Wampler 90] are purely numerical procedures able to find all possible solutions of the set of non-linear equations (in contrast to other numerical methods, such as Newton- Raphson, which converge to a single solution). These methods are based on homotopy techniques for gradually transforming a ``start'' system whose solutions are known to the system whose solutions are sought. These methods are robust and applicable to any set of equations. However, since they are iterative procedures, they are too slow in practice. Elimination approaches use one of the next algebraic methods: the Gr\"obner Basis method [Buchberger 82], which is an iterative variable elimination technique, or the resultant method [Gelfand 94], capable of eliminating all but one variable in a single step. In both cases, the elimination process normally leads to an univariate polynomial, relatively easy to solve [Pan 99]. The applicability of the Gr\"obner Basis method is mainly limited by its algorithmic complexity. Resultant methods can provide computationally fast techniques, but they require geometric intuition to nd (if possible) the formula for the resultant.

\subsubsection{Alternative methods}

Lately, interval methods for solving systems of non-linear equations have been proposed
as an alternative to continuation and elimination methods. They are based on interval
arithmetic [Moore 79, Hansen 92] and manipulate upper and lower bounds of variables.
Two main classes of interval-based methods have been applied in Robotics: those based
on the interval version of the Newton method [Rao 98, Castellet 98], and those based
on subdivision [Sherbrooke 93, Merlet 01, Porta 02]. They are completely numerical and
robust techniques. Although implemented techniques are still slow, recent improvements
are signifficantly increasing their efficacity [Porta 03].

\subsection{Geometric substitution}

We only treat here the geometric substitution case, with some simple example to give a broad insight of the approach. The geometric methods are not general, in the sense that they cannot be applied automatically but rather require the intuition of a robotic engenieer to chose the proper substitution. These methods were generalized in the case of 6-DOF revolute robots: the subsequent algebraic substitution are general (they work automatically for any revolute 6-DOF robots) but have not been generalized to other kinematic chains. 

\subsubsection{Simple planar 2R robot}

A 2R robot is a fixed kinematic chain composed of two revolute joints with parallel rotation axis. The configuration space is $\mathbb{S}^1  \times \mathbb{S}^1$. It therefore evoles in the plane orthogonal to the rotation axes. Two axes does not enable to control at the same time the two positions and the orientation. We therefore only consider the projection of the direct geometry function in the position space $\mathbb{R}^2$ (removing the orientation). The projected direct geometry map is:
\[ K: (q_1,q_2) \in \mathbb{S}^1  \times \mathbb{S}^1 \rightarrow \Big( l_1 cos(q_1)+ l_2 cos(q_1+q_2),l_1 sin(q_1)+ l_2 sin(q_1+q_2) \Big) \]
where $l_1$,$l_2$ are the length of the two bodies of the robot.

Now, considering a point $p \in \mathbb{R}^2$, the inverse geometry returns the configuration reaching this point. We denote $\rho = || p ||$ the distance of $p$ to the robot basis and $\alpha = \frac{1}{\rho} e_1^T p$ the angle of the direction from the robot basis to $p$.

By anthropomorphism, the first joint is called the ``shoulder'' and the second the ``elbow''. The configuration of the elbow is directly given by the distance from the robot to the point:
\[ q_2 = \]

The shoulder angle is given by the angle to the point:
\[ \]

\subsubsection{And the wrist?}

The same decomposition can be applied if two shoulder joints drives the robot orientation in the tridimensionnal space. 

If a set of joints are attached to the end effector to enable the control of the orientation, a third subproblem should be solved, to map the end effector desired orientation to the wrist angles. In that case, we first solve the wrist orientation, then the elbow lengthening and finally the reach orientation with the shoulder.

A more formal and very complete description of the geometrical substitution can be found in the book of [Murray94].


% ------------------------------------------------------------------------------
% ------------------------------------------------------------------------------
% ------------------------------------------------------------------------------
\section{Iterative optimization} \label{sec:numeric}

In the previous section, we have defined the inverse geometry problem and shown that there is no solution (yet) to solve it algebrically in the general case. Therefore, the only remaining solution is to approximate it numerically.

The inverse geometry problem comes back to find the root of a non-linear locally convex real-value function. Since $K$ is not real-value and is not null at the resolution of the problem, we have to define such a function to encapsulate our problem, for example:
\[ f: q \in \mathcal{Q}  \rightarrow \textrm{dist}(K(q),m^*)  \]
where $m^*$ is the reference position of the end effector in $SE(3)$ and dist is a distance (to be properly chosen) in $SE(3)$. Since $m^*$ might be out of the workspace, the function might never vanish. If such an instance of the problem might be implemented, it might be more interesting to search for the global optimum of the function rather than for its root (that will match, should the function vanishes). The inverse geometry problem is then rewritten as an optimization problem
\[  \min_{q \in \mathcal{Q}} f(q) \]

In this section, we will introduce the basics of the methods to solve this kind of problem. To simplify the presentation, we will focuse on the optimization in $mathbb{R}^n$, \mie in a given global coordinates system rather than directly in the manifold. Care has to be taken when coming back to more general manifolds (like $SE(3)$) to properly take into account the topology. We will focuse here on the optimization algorithms rather than on the analysis of the optimization problem itself (optimality conditions, caracterization, etc). A complete introduction to the optimization analysis can be found in [Hiriart-Urruty 11].

\subsection{Optimality condition}

We consider here only twice-differentiable $\mathcal{C}^2$ functions from a vector space $\mathbb{R}^n$ to $\mathbb{R}$:
\[ f: \mathbb{R}^n \rightarrow \mathbb{R} \]

\begin{definition}[Global minimizer]
The global minizer of $f$ is a vector $x^*\in\mathbb{R}^n$ if :
\[ \forall x \in \mathbb{R}^n, f(x^*) < f(x) \]
The minimum of $f$ is $f(x^*)$.
\end{definition}

Such a vector $x^*$ may not exist. Typically, the function might not be lower bounded, of the lower bound ($inf$ on the function) may not be reached or the minimum can be reached in several vector, possibly defining an implicit submanifold on $\mathbb{R}^n$.

Without any further hypothesis on the geometry of $f$, we do not have any condition to characterize $x^*$. We are reduce to study the local minization conditions.

\begin{definition}[Local (strict) minimizer]
A local strict minizer of $f$ is a vector $x^*\in\mathbb{R}^n$ such that there exist a neighborghood of $\mathcal{N}$ of $x^*$ where:
\[ \forall x \in \mathcal{N}, f(x^*) < f(x) \]
The value $f(x^*)$ is named the local minimum of $f$.
\end{definition}

Take care that the neighborghood can very well contain another local minimizer, \mie that there is a sequence of local minimizers of $f$ converging to $x^*$ (typically for functions that oscillates with diverging frequency when approaching to $x^*$.

The local minimization can intuitively be characterized by the derivatives of $f$. For additional conditions, it might be possible to guarantee that there exist only one local minimizer, \mie that a local minimizer is a global minimizer: if $f$ is strictly convex, any local minimizer is the global minimizer. Similarly, if we reduce the search to an open set of $\mathbb{R}^n$ where $f$ is convex, then a local minimizer on this set is a global minizer on this set. This is why the optimization methods presented below are often named ``convex optimization''.

A local minimizer is characterized by the following conditions:

\begin{theorem}[First order necessary condition]
  If $x^*$ is a local minimizer of $f$ \emph{\underline{then}} the gradient vanishes at $x^*$: \[\dpartial{f}{x}(x^*) = 0\].
\end{theorem}

The reciprocal is of course wrong: in general, a point where the gradient vanishes is a static point of $f$ (it might typically be a saddle point). The Hessian matrix can be used to distinguish between minimizers and static points.

\begin{theorem}[Second order necessary condition]
If $x^*$ is a local minimizer of $f$ \emph{\underline{then}} $\dpartial{f}{x}(x^*) = 0$ and the Hessian matrix is positive semidefinite: \[ \dpartial{f}{x}(x^*) = 0, \quad \ddpartial{f}{x}(x^*) \ge 0 \]
\end{theorem}

Here the Hessian is only guaranteed to be semidefinite.
This condition also holds for local non-strict minimizers, \mie if $x^*$ only satisfy $f(x^*) \le f(x)$ on a neighborghood. If the Hessian is definite positive it then characterized a necessary condition. 

\begin{theorem}[Second order necessary condition]
The vector $x^*$ is a local minimizer of $f$ \emph{\underline{if}} $\dpartial{f}{x}(x^*) = 0$ and the Hessian matrix is positive definite: \[ \dpartial{f}{x}(x^*) = 0, \quad  \ddpartial{f}{x}(x^*) > 0 \]
\end{theorem}

These three conditions can be easily demonstrated using the Taylor development of $f$, recalled in the following theorem.
\begin{theorem}[Taylor development]
Consider $f\in\mathcal{C}^2$. Then the developments of $f$ in $x$ are:
\begin{itemize}
\item (first order development) for any $p\in \mathbb{R}^n$, there exists $t\in\mathbb{R}$ such that:
\[ f(x+p) = f(x) + \dpartial{f}{x}(x+tp) p \]
\item (second order development) for any $p\in \mathbb{R}^n$, there exists $t\in\mathbb{R}$ such that:
\[ f(x+p) = f(x) + \dpartial{f}{x}(x) p + \frac{1}{2} p^T \ddpartial{f}{x}(x+tp)p  \]
\end{itemize}
\end{theorem}

\subsection{Overview of the algorithms}

The optimality conditions only use the information provided by the derivative of $f$. Similarly, the optimization algorithm will use this same information to search for the solution. 

All the algorithms start with an initial vector $x_0$, called the initial guess, and will iteratively improve this guess until it finally converges to a local optimum $x^*$. At each iteration $k$, all the algorithm roughly use the same schema: they first decide a direction of descent, \mie a direction $p_k$ of (tangent vector to) $\mathbb{R}^n$ where the next guess will be search; then the next vector $x_{k+1}$ is searched in this direction by choosing a step length $\alpha \in \mathbb{R}^+$ by solving the one-dimensionnal problem on the variable $\alpha$:
\EIN{stepmin} \min_{\alpha>0} f(x_k+\alpha p_k) \EOUT
The direction $p_k$ is chosen from a local model of $f$ around $x_k$, typically build using the derivative of $f$ at $x_k$, or using the derivatives at the previous $x_i$, $i\le k$. The step length $\alpha_k$ might be fixed (arbitrarily chosen from an apriori canonical unit of the system) ; or computed by dichotomy (line search algorithm) ; or chosen from the local model of the function (trust region algorithm).

The  iterative descente methods build a sequence of points $\Big(x_k \Big)_{k\in\mathbb{N}}$ with $\forall k, f(x_k) \le f(x_{k+1})$. Under some decrease conditions, the sequence converges to a local minimizer $x^*$. The convergence is assymptotic, which means that in a finite amount of time, only an approximation of $x^*$ is finally obtained in general. From the decrease conditions, a characterization of how quickly the algorithm converges toward the local minimizer can also be obtained.

\subsection{Descent direction}

\subsubsection{Gradient descent method}

The first method is to choose the descent direction given by the gradient of $f$ in $x_k$:
\[ p_k = -\frac{1}{|| \nabla f_k||} \nabla f_k \] 
with $\nabla f_k \triangleq \dpartial{f}{x}(x_k)$ the gradient of $f$.

The gradient is indeed the speepest descent direction. Indeed, for a direction $p$ unitary and a step length $\alpha$, the rate of change in the direction $p$ is given  by the derivative of $\alpha \rightarrow f(x_k+\alpha p)$. Using the Taylor second order development:
\[ f(x_k +\alpha p) = f(x_k) + \alpha p^T \nabla f_k + \frac{1}{2} \alpha^2 p^T \ddpartial{f}{x}(x_k+tp) p \]
for some $t\in\mathbb{R}$. The derivative of this function for $\alpha=0$ should be minimize for maximizing the descent, which directly gives $p^* = - \nabla f_k$.

The gradient is a proper direction descent, in the sense that for $\alpha$ sufficiently small, it decreases the value of $f$:
\[ \exists \alpha>0, f(x_k+\alpha \nabla f_k) < f(x_k) \]
This is prove using similar arguments than in the previous statement.

The gradient method converges linearly. This is stated by the following theorem.
\begin{theorem}[Rate of convergence of the gradient descent]
Consider a sequence $(x_k)_{k\in\mathbb{N}}$ generated by a gradient descent on $f$ with exact line search (\mie the minimum of \eq{eq:stepmin} is exactly reached for any $f$) converging to a point $x^*$ where the Hessian of $f$ is positive definite. Then, their exist $0<\rho<1$ and $N\in\mathbb{N}$ such that:\[ \forall k>N, f(x_{k+1}) < \rho f(x_k) \]
The rate $\rho$ can be computed from the eigenvalues of the Hessian:
\[ \rho = \Big( \frac{\lambda_n - \lambda_1}{\lambda_n + \lambda_1} \Big)^2 \]
\end{theorem}


\subsubsection{Newton method}

\subsubsection{Quasi-Newton method}

\subsubsection{Gauss-Newton method}


\subsection{Step length}


% ------------------------------------------------------------------------------
% ------------------------------------------------------------------------------
% ------------------------------------------------------------------------------
\section{Numerical inverse geometry}\label{sec:numinvgeom}

% ##############################################################################
% ##############################################################################
% ##############################################################################
\chapter{Inverse kinematics}

\section{The pseudo inverse}

\section{The robot Jacobian}

\section{Inverting the Jacobian}

\section{Redundancy}

\section{Active-set search}

% ##############################################################################
% ##############################################################################
% ##############################################################################
\chapter{Inverse dynamics}

\section{Velocities and forces}

\section{The dynamic equation of a multi#body system}

\subsection{Recursive Newton#Euler algorithm}

\section{Operational#space inverse dynamics}

\section{Synthesis: a overview of constrained optimization}

% ##############################################################################
% ##############################################################################
% ##############################################################################
\chapter{Optimal control}

\section{Inifinite#space optimization}

\section{Discretization and static optimization}

\section{Linear#quadratic resolution}

\section{The example of the robot walk}


\end{document}
